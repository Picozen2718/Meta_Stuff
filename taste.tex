\section*{Taste:}
The objective is to move beyond passive reading to critical evaluation of Theoretical Computer Science (TCS) research.

\begin{itemize}
    \item \textbf{Read the papers accepted at SODA/FOCS/STOC} $\rightarrow$ The goal here is to note the authors, what kinda + quality of the ideas, just know what's going on \& who's doing it. No need to get the whole idea \& detail but you must get the people, problem, source of hardness, main techniques, why do you think it was selected.
    
    \item \textbf{Read papers from ICALP, ESA, STACS, CCC, WADS, APPROX, RANDOM, ISAAC} $\rightarrow$ here aim to understand the main ideas to such a degree that you can start to see where they are headed. Approach in a way to guess and actually build with the authors. If any paper is actually really interesting, read it fully. \\
    After that look for other ideas you would have tried and why they made those choices $\rightarrow$ what will \& will not work! Then see if there are any open questions $\rightarrow$ try one or two. \\
    Note the authors of the papers you enjoyed. \\
    For these papers, it's important you understand why it was selected \& imagine yourself writing it $\rightarrow$ \textit{image training is most important here!} (Same principle for when you attend conferences).
    
    \item Eventually once this is comfortable for you start browsing \textbf{arXiv} daily too.
    
    \item \textbf{It is important you feel like you could write them.} Atleast think about what you would do \& how would you approach them constantly as you read. The start asking questions as you read. Try to predict the direction this is going. \\
    Take your time but ask good questions \& have your answer, even if it's a dumb answer, have an answer as you proceed reading. \\
    Taste will eventually be how good are your questions, your instinct on each question, and your exposure to bigger ideas $\rightarrow$ knowing what's a good paper at each level!
\end{itemize}