\section*{Technical Toolkit:}
\begin{itemize}
    \item There are first the basics of an undergrad degree. The core courses. You must atleast know this much. In math \& CS. Use GATE / ETH acad reqs / ISI / CMI / IISc curriculum for this. No need to stress a lot about electives but atleast the core courses should be great for you $\rightarrow$ make temp notes if you need but these are things mostly in your memory $\rightarrow$ maybe digital notes in this case/mindmaps will be good.
    
    \item Then there are topics needed for you. Formal courses in Algorithms, Complexity, etc. This goes beyond completing coursework. You need to understand this from every angle, like how you know school topics. Make notes, solve every question, watch lectures, go all in here. \\
    Sir had suggested picking one book and solving it fully $\rightarrow$ Do that!
    
    \item Topics here are: (1) Algorithms, (2) Approximation, (3) Parameterized, (4) Randomized, (5) Geometric, (6) Complexity.
    
    \item The goal here is to develop a toolkit. What techniques are you most comfortable with? What approach is intuitive to you? Cause at the end of the day the problems are the same, your toolkit is what changes! So, 7-8 techniques this sem is also good, but you should be able to use it.
    
    \item Try and apply these to the papers \& open problems you're looking at.
    
    \item Eventually, you'll develop a \underline{style} and go more \& more deep in that. $\rightarrow$ Hopefully in 2 years it'll be noticeable!
\end{itemize}